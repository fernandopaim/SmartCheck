\documentclass[9pt]{sigplanconf}
\usepackage{url}
\usepackage{tikz}
\usepackage{graphicx}
\usepackage{framed}
\usepackage{color}
\usepackage{url}
\usepackage{alltt}
\usepackage{tabularx}
\usepackage{subfigure}
\usepackage{fancyhdr}
\usepackage{paralist}

\usepackage{pgf}
%% \usepackage{tikz}
%% \usetikzlibrary{arrows,automata}

%% \usepackage{amsmath}
%% \usepackage{amsthm}

%% \newtheoremstyle{lst}
%%   {\topsep}   % ABOVESPACE
%%   {\topsep}   % BELOWSPACE
%%   {\itshape}  % BODYFONT
%%   {0pt}       % INDENT (empty value is the same as 0pt)
%%   {} % HEADFONT
%%   {}         % HEADPUNCT
%%   {5pt plus 1pt minus 1pt} % HEADSPACE
%%   {(\thmnumber{#2})}          % CUSTOM-HEAD-SPEC
%% \theoremstyle{lst}
%% \newtheorem{codenum}{lst}

\usepackage{float}
\floatstyle{boxed} 
\restylefloat{figure}

%% \include{math}

\newenvironment{code}{\begin{alltt}\scriptsize}{\end{alltt}}
%% \newenvironment{cols}{\begin{tabular}{m{3.6cm}|m{3.6cm}}{Haskell} &
%%     {\scriptsize Copilot}\\\hline}{\end{tabular}}

\newcommand{\ttp}[1]{\texttt{#1}}

\usepackage{ifthen}
\newboolean{submission}  %set to true for the submission version
%\setboolean{submission}{false}
\setboolean{submission}{true}
\ifthenelse
{\boolean{submission}}
{ \newcommand{\todo}[1]{ } } %hide todo
{ \newcommand{\todo}[1]{ {\color{blue}$<<$#1$>>$}
 }}
%\usepackage{fancyhdr}


\begin{document}

%% \conferenceinfo{ICFP'12,} {September 9--15, 2012, Copenhagen, Denmark.}
%% \CopyrightYear{2012}
%% \copyrightdata{978-1-4503-1054-3/12/09}

%\titlebanner{banner above paper title}        % These are ignored unless
%\preprintfooter{}   % 'preprint' option specified.

\title{SmartCheck}
%\subtitle{Subtitle Text, if any}

\authorinfo{Lee Pike}
           {Galois, Inc.}
           {leepike@galois.com}
\maketitle

\begin{abstract}
foobar
\end{abstract}

%% \category{D.2.4}{Software/Program Verification}{Reliability}

%% \terms
%% Languages, Verification

%% \keywords
%% embedded domain-specific language, compiler, verification


\section{Introduction}\label{sec:intro}

The QuickCheck testing framework was a revolutionary step-forward in
type-directed random testing~\cite{qc}.  Nowadays, QuickCheck is a standard
aspect of the Haskell development cycle.  

Because QuickCheck generates
random values for testing, counterexamples it finds may not be substantially
larger than a minimal counterexample.  In their original QuickCheck
paper~\cite{qc}, the authors report the following user experience by Andy Gill:
%
\begin{quote}
Sometimes the counter examples found are very large and it is difficult to go
back to the property and understand why it is a counter example.
\end{quote}
%
\noindent
QuickCheck defines a type class \ttp{Arbitrary} that defines a function
\ttp{arbitrary} for generating random values of a given type.  Gill added another
method to the type class:
%
\begin{code}
smaller :: a -> [a] 
\end{code}
%
\noindent
The purpose of \ttp{smaller} is to generate strictly smaller values (according to
some measure) from a given counterexample to attempt to find a smaller
counterexample.  Today, \ttp{smaller} is called \ttp{shrink}.

\begin{figure}[ht]
\begin{code}
data A = A Int16 deriving Show

data B = B [A] [A] [A] [A] deriving Show

-- QuickCheck instances
instance Arbitrary A where
  arbitrary    = liftM A arbitrary
  shrink (A i) = map A (shrink i)

instance Arbitrary B where
  arbitrary = liftM4 B arbitrary arbitrary arbitrary arbitrary
  shrink (B a b c d) = 
    [ B w x y z | w <- shrink a, x <- shrink b
                , y <- shrink c, z <- shrink d ]

-- SmallCheck instances
instance Serial Int16 where
  series d = drawnFrom [(-d')..d'] where d' = fromIntegral d

instance Serial A where series = cons1 A 
instance Serial B where series = cons4 B

-- Predicates for defining properties
pre :: B -> Bool
pre (B a b c d) = and \$ map pre' [a, b, c, d]
  where pre' x = add x < 16
  
post :: B -> Bool
post (B a b c d) = add a + add b + add c + add d < 64
  where add :: [A] -> Int16
        add = sum . map (\(\backslash\)(A i) -> i)

-- Property
prop :: B -> Property
prop p = pre p ==> post p
\end{code}
  \caption{QuickCheck and SmallCheck for a product type input.}
  \label{fig:initial}
\end{figure}

While \ttp{shrink} is user-defined and can have any type-correct implementation,
it is typically defined using structural recursion for composite data types.
For example, consider the \ttp{shrink} instances given to data types \ttp{A} and
\ttp{B} in Figure~\ref{fig:initial}.  Indeed, the principal motivation for one
of Haskell's first generics package was for providing generic definitions for
\ttp{shrink}~\cite{syb}.

However, this straightforward approach to defining \ttp{shrink} instances may
either miss minimal counterexamples, be too inefficient to be practical, or
both.\footnote{E.g., see the user question posed on StackOverflow:
  \url{http://stackoverflow.com/questions/8788542/how-do-i-get-good-small-shrinks-out-of-quickcheck}.}
For example, consider again Figure~\ref{fig:initial}.  The data type \ttp{B} is
a product type over lists of 16-bit signed integers, wrapped by a constructor.
Consider the property \ttp{prop} in the figure, stating that if the sum of each
list in a field of \ttp{B} is less than 16, then the sum of the four fields of
\ttp{B} is less than 64.  \ttp{prop} seems reasonable at first glance, until
one realizes that due to underflow, the property can be violated.  For example,
consider the value
%
\begin{code}
B [A (-32769)] [] [] []
\end{code}
%
\noindent
Without shrinking, QuickCheck finds large examples to the property; on average,
a value contains 70 \ttp{Int16} values!

Thus, it pays to define \ttp{shrink}.  Unfortunately, even for a simple property
and program like this one, the definition for \ttp{shrink} given in
Figure~\ref{fig:initial} produces an intractable number of potential
counterexamples to test; with the definitions and property provided, which also
rely on QuickCheck's default instances for the list type, the list of potential
counterexamples generated by \ttp{shrink} can contain more than $10^{10}$
elements.  This is an enormous state space to search for a smaller example!

It is not always straightforward to determine how to make a more efficient
implementation for \ttp{shrink}.  Shrinking presents a programmer's dilemma:
shrinking is used to better understand a counterexample to understand why a
property failed; but to implement an efficient shrink function in general
requires knowing how to generate counterexamples to the property.

Typically, a user tries a brute-force approach to reduce the number of potential
counterexamples, for example by truncating lists using the \ttp{take n} function
that returns the first $n$ elements of a list.  The trade-off is quicker
shrinking with a lower probability of finding a smaller counterexample.  For
example, redefining shrink for type \ttp{B} as follows
%
\begin{code}
shrink (B a b c d) = [ B w x y z | w <- tk a, x <- tk b
                                 , y <- tk c, z <- tk d ]
  where tk x = take 10 (shrink x)
\end{code}
%
\noindent
controls the exponential blowup of the state-space by taking the first ten
elements of each list, so the cross product is limited to 1000 elements.  The
downside is that potentially smaller counterexamples may be omitted.

What if we omit the need for shrinking counterexamples altogether?  SmallCheck
is another testing framework for Haskell.  SmallCheck is guaranteed to return
the a smallest counterexample, if one exists~\cite{sc}.  SmallCheck does this by
enumerating all possible inputs, ordered from smallest to largest, up to some
user-defined bound.  While SmallCheck is effective for testing many programs and
properties (in accordance with the \emph{small scope hypothesis}~\cite{jackson}),
counterexamples to even relatively simple properties may be infeasible to
discover due to state-space explosion.

With SmallCheck, a property is checked up to a depth set by the user, measuring
how large values are.  For example, for type \ttp{Int}, the depth $d$ is defined
as the integers in the list enumerating from $-d$ to $d$ (i.e.,
\ttp{[(-d)..d]}).  For a product type, SmallCheck must check values produced by
taking the cross-product of the type's fields at a given depth.  In
Figure~\ref{fig:initial}, we have defined instances for generating Lazy
SmallCheck (a more efficient version of SmallCheck) tests by defining instances
of the \ttp{Serial} class.  For the property \ttp{prop} defined above (suitably
redefined to use SmallCheck's types), SmallCheck must check to depth 16 to find
the first counterexample.  Unfortunately, after a couple hours of testing, Lazy
SmallCheck is still checking values at depth six, and the number of tests scales
exponentially with respect to the depth.\footnote{All tests in this paper are
  performed on a four-core Pentium~i7 running at 2.7GHz with 8GB RAM on
  Fedora~16.  This test, and others unless noted, are performed using
  interpreted Haskell under GHC 7.4.2.}

Finally, even if we could overcome the problems described above with QuickCheck
and SmallCheck, we are left with two others.  First, when a
counterexample is discovered, one might wish to search for different
counterexamples.  There currently is no automated way to add a predicate
characterizing a counterexample as preconditions to a property to strengthen the
property.  Second, and more generally, a counterexample is usually
representative of other counterexamples, but that generalization is left to the
user to discover for herself.  Really, what would be useful would be not just a
minimal counterexample, but a quantified formula characterizing a set of
counterexamples.  Such a formula has the added benefit of replacing a
potentially large sub-term of a composite data type value with a quantifier.


\begin{figure}[ht]
\scriptsize
  \begin{center}
    \begin{tabular}{|r||c|c|c|c|}
\hline 
 & QC (none) & QC (10) & QC (20) & SmartCheck \\      
\hline \hline 
Max.  & 0.15s & 21.51s & 125.37s & 1.96s\\
\hline
Mean  & 0.07s & 1.21s & 3.80s & 0.30s\\
\hline
Median & 0.07s & 0.48s & 0.52s & 0.24s\\
\hline
Mean size & 70 & 34 & 32 & 5\\
\hline
    \end{tabular}
  \end{center}
  \caption{Results for 100 tests of the program in Figure~\ref{fig:initial}
    using interpreted Haskell.}
  \label{fig:results}
\end{figure}


\paragraph{SmartCheck}
Motivated by these limitations of QuickCheck and SmallCheck, we have developed
\emph{SmartCheck}.  SmartCheck extends QuickCheck to efficiently and generically
find small counterexamples.  To motivate the benefit of SmartCheck, consider the
table in Figure~\ref{fig:results}.  The table characterizes testing the program
shown in Figure~\ref{fig:initial} that we have described so far.  Four runs are
presented.  In the first three, we use QuickCheck to generate counterexamples.
The first column contains the results of using QuickCheck without shrinking to
provide a baseline.  The next two columns show the result of using QuickCheck
with definitions of \ttp{shrink} that limit the lists of fields of \ttp{B} to 10
(the ``QC (10)'' tests) and 20 (the ``QC (20)'' tests) elements, respectively,
using a definition like the truncated implementation of \ttp{shrink} for the
type \ttp{B} discussed earlier.  In each row, we show the maximum, mean, and
median number of seconds required to by QuickCheck to generate a counterexample
over 100 runs.  In the last row, we show the mean size of the final value
returned, measured by the taking the cumulative length of the lists of the
fields of \ttp{B}.  Notice that while typically counterexamples are returned in
a fraction of a second, there are outliers taking 21 seconds and 125 seconds for
the two tests.  Both tests roughly cut the size of the counterexample in half;
increasing the number of potential counterexamples can result in a significant
performance penalty for this example without much benefit.

In the last column are the results under the same condition for SmartCheck.  Not
only does it execute faster than QuickCheck with shrinking, it has significantly
smaller outliers.  The most striking difference, however, is the size of the
counterexample returned, reducing on average the size of a counterexample found
by QuickCheck without shrinking by a factor of 14.

The purpose of this paper is to explain first how to efficiently and generically
generate small counterexamples for algebraic data types and second how to
generalize the counterexamples..

\paragraph{Contributions}
This paper, and the library it describes, makes the following contributions:

\begin{itemize}

\item Using generic programming techniques, we describe efficient counterexample
  reduction strategies for algebraic data types.

\item We describe an approach to counterexample generalization, to present the
  user with a formula describing a set of counterexamples to a property.

\item We present an approach to generically strengthen a property with a
  precondition that characterizes an already-discovered counterexample for the
  purpose of finding new, dissimilar counterexamples.

\item More generally, this paper is the first to explore the idea of efficient,
  generic counterexample reduction and generalization for functional program testing.
\end{itemize}

\paragraph{Outline.}


%%%%%%%%%%%%%%%%%%%%%%%%%%%%%%%%%%%%%%%%%%%%%%%%


\newpage

\todo{insight: shrink based on arbitrary}


\section{Shrinking Data}


\section{Generalizing Data}


\section{Implementation}


\section{Experiments}

%--------------------------------------------------------------------------------


%--------------------------------------------------------------------------------
\section{Related Work}
\label{sec:related}

\todo{\cite{syb} paper is motivated by generic shrink}
\todo{shrink paper rehger pointed to ``Automatic isolation of compiler errors'' from 1994}

\cite{}

%--------------------------------------------------------------------------------
\section{Conclusions}
\label{sec:conclusions}


\section*{Acknowledgements}
\todo{Thank Rehger for comments/pointing to paper.}

%% \balancecolumns

\bibliographystyle{abbrvnat}
\bibliography{paper}

\end{document}

