% can be 9pt
\documentclass[10pt]{sigplanconf}
\usepackage{url}
\usepackage{tikz}
\usepackage{graphicx}
\usepackage{framed}
\usepackage{color}
\usepackage{url}
\usepackage{alltt}
\usepackage{tabularx}
\usepackage{subfigure}
\usepackage{fancyhdr}
\usepackage{paralist}

\usepackage{pgf}

\usepackage{listings}
\lstset{ %
  language=Haskell,                 % the language of the code
  %%basicstyle= \scriptsize,           % the size of the fonts that are used
  %% for the code
  %%firstnumber=0, % index to start numbering at
  %%numbers=left,                      % where to put the line-numbers
  %% numberstyle=\tiny\color{gray},  % the style that is used for the
  %% line-numbers
  %% stepnumber=5,                   % the step between two line-numbers. If it's
                                  % 1, each line 
  %%                              % will be numbered
  %% numbersep=5pt,                  % how far the line-numbers are from the code
  %% xleftmargin=0ex,
  %% backgroundcolor=\color{white},      % choose the background color. You must
  %% add \usepackage{color}
  %% showspaces=false,               % show spaces adding particular underscores
  %% showstringspaces=false,         % underline spaces within strings
  %% showtabs=false,                 % show tabs within strings adding
  %% particular underscores
  %% frame=single,                   % adds a frame around the code
  %% rulecolor=\color{black},        % if not set, the frame-color may be
  %% changed on line-breaks within not-black text (e.g. commens (green here))
  %% tabsize=2,                      % sets default tabsize to 2 spaces
  %% captionpos=b,                   % sets the caption-position to bottom
  %% breaklines=true,                % sets automatic line breaking
  %% breakatwhitespace=false,        % sets if automatic breaks should only
  %% happen at whitespace
  %% title=\lstname,                   % show the filename of files included
  %% with \lstinputlisting;
  %%                                 % also try caption instead of title
  %% keywordstyle=\color{blue},          % keyword style
  %% commentstyle=\color{dkgreen},       % comment style
  %% stringstyle=\color{mauve},          % string literal style
  escapeinside={@*}{*@},            % if you want to add a comment within your
                                    % code
  %% morekeywords={*,...}           % if you want to add more keywords to
  %% the set
}


%% \usepackage{tikz}
%% \usetikzlibrary{arrows,automata}

%% \usepackage{amsmath}
%% \usepackage{amsthm}

%% \newtheoremstyle{lst}
%%   {\topsep}   % ABOVESPACE
%%   {\topsep}   % BELOWSPACE
%%   {\itshape}  % BODYFONT
%%   {0pt}       % INDENT (empty value is the same as 0pt)
%%   {} % HEADFONT
%%   {}         % HEADPUNCT
%%   {5pt plus 1pt minus 1pt} % HEADSPACE
%%   {(\thmnumber{#2})}          % CUSTOM-HEAD-SPEC
%% \theoremstyle{lst}
%% \newtheorem{codenum}{lst}

\usepackage{float}
\floatstyle{boxed} 
\restylefloat{figure}

%% \include{math}

\newenvironment{code}{\begin{alltt}}{\end{alltt}}
%% \newenvironment{cols}{\begin{tabular}{m{3.6cm}|m{3.6cm}}{Haskell} &

\newcommand{\ttp}[1]{\texttt{#1}}

\usepackage{ifthen}
\newboolean{submission}  %set to true for the submission version
%\setboolean{submission}{false}
\setboolean{submission}{true}
\ifthenelse
{\boolean{submission}}
{ \newcommand{\todo}[1]{ } } %hide todo
{ \newcommand{\todo}[1]{ {\color{blue}$<<$#1$>>$}
 }}
%\usepackage{fancyhdr}

\newcommand{\sub}[1]{\(\sb{#1}\)}

\begin{document}

%% \conferenceinfo{ICFP'12,} {September 9--15, 2012, Copenhagen, Denmark.}
%% \CopyrightYear{2012}
%% \copyrightdata{978-1-4503-1054-3/12/09}

%\titlebanner{banner above paper title}        % These are ignored unless
%\preprintfooter{}   % 'preprint' option specified.

\title{SmartCheck}
%\subtitle{Subtitle Text, if any}

\authorinfo{Lee Pike}
           {Galois, Inc.}
           {leepike@galois.com}
\maketitle

\begin{abstract}
foobar
\end{abstract}

%% \category{D.2.4}{Software/Program Verification}{Reliability}

%% \terms
%% Languages, Verification

%% \keywords
%% embedded domain-specific language, compiler, verification


\section{Introduction}\label{sec:intro}

The QuickCheck testing framework was a revolutionary step-forward in
type-directed random testing~\cite{qc}.  Nowadays, QuickCheck is a standard
aspect of the Haskell development cycle.  Because QuickCheck generates random
values for testing, counterexamples it finds may not be substantially larger
than a minimal counterexample.  In their original QuickCheck paper~\cite{qc},
the authors report the following user experience by Andy Gill:
%
\begin{quote}
Sometimes the counterexamples found are very large and it is difficult to go
back to the property and understand why it is a counterexample.
\end{quote}
%
\noindent
QuickCheck defines a type class \ttp{Arbitrary} that presents a method
\ttp{arbitrary} for generating random values of a given type.  Gill added another
method to the type class:
%
\begin{code}
smaller :: a -> [a] 
\end{code}
%
\noindent
The purpose of \ttp{smaller} is to generate strictly smaller values (according to
some measure) from a given counterexample to attempt to find a smaller
counterexample.  Today, \ttp{smaller} is called \ttp{shrink}.

\begin{figure}[ht]
\begin{code}
data A = A Int16 deriving Show

data B = B [A] [A] [A] [A] deriving Show

-- QuickCheck instances
instance Arbitrary A where
  arbitrary    = liftM A arbitrary
  shrink (A i) = map A (shrink i)

instance Arbitrary B where
  arbitrary = liftM4 B arbitrary arbitrary 
                       arbitrary arbitrary
  shrink (B a b c d) = 
    [ B w x y z | w <- shrink a
                , x <- shrink b
                , y <- shrink c
                , z <- shrink d ]

-- SmallCheck instances
instance Serial Int16 where
  series d = drawnFrom [(-d')..d'] 
    where d' = fromIntegral d

instance Serial A where series = cons1 A 
instance Serial B where series = cons4 B

-- Predicates for defining properties
pre :: B -> Bool
pre (B a b c d) = and \$ map pre' [a, b, c, d]
  where pre' x = add x < 16
  
post :: B -> Bool
post (B a b c d) = 
  add a + add b + add c + add d < 64
  where add :: [A] -> Int16
        add = sum . map (\(\backslash\)(A i) -> i)

-- Property
prop :: B -> Property
prop p = pre p ==> post p
\end{code}
  \caption{QuickCheck and SmallCheck for a product-type input.}
  \label{fig:initial}
\end{figure}

While \ttp{shrink} is user-defined and can have any type-correct implementation,
it is typically defined using structural recursion for composite data types.
For example, consider the \ttp{shrink} instances given to data types \ttp{A} and
\ttp{B} in Figure~\ref{fig:initial}.  Indeed, because of the regular structure
of \ttp{shrink} implementations, the principal motivation for one
of Haskell's first generics package was for providing generic definitions for
\ttp{shrink}~\cite{syb}.

However, this straightforward approach to defining \ttp{shrink} instances may
either miss minimal counterexamples, be too inefficient to be practical, or
both.\footnote{E.g., for a recent real-world case, see the user question posed on StackOverflow:
  \url{http://stackoverflow.com/questions/8788542/how-do-i-get-good-small-shrinks-out-of-quickcheck}.}
For example, consider again Figure~\ref{fig:initial}.  The data type \ttp{B} is
a product type over lists of 16-bit signed integers, wrapped by a constructor.
Consider the property \ttp{prop} in the figure, stating that if the sum of each
list in a field of \ttp{B} is less than 16, then the sum of the four fields of
\ttp{B} is less than 64.  \ttp{prop} seems reasonable at first glance, until
one realizes that due to underflow, the property can be violated.  For example,
consider the value
%
\begin{code}
B [A (-32769)] [] [] []
\end{code}
%
\noindent
Without shrinking, QuickCheck finds large examples to the property; on average,
QuickCheck returns a value contains 70 \ttp{Int16} values!

Thus, it pays to define \ttp{shrink}.  Unfortunately, even for a simple property
and program like this one, the definition for \ttp{shrink} given in
Figure~\ref{fig:initial} produces an intractable number of potential
counterexamples to test; with the definitions and property provided, which also
rely on QuickCheck's default instances for the list type, the list of potential
counterexamples generated by \ttp{shrink} can contain more than $10^{10}$
elements.  This is an enormous state space to search for a smaller example!  In
particular, if a counterexample is not found does not finish in a reasonable
amount of time.

It is not always straightforward to determine how to make a more efficient
implementation for \ttp{shrink}.  Shrinking presents a programmer's dilemma:
shrinking is used to better understand a counterexample to understand why a
property failed; but to implement an efficient shrink function in general
requires knowing how to generate counterexamples to the property.

Typically, a user tries a brute-force approach to reduce the number of potential
counterexamples, for example, by truncating lists using the \ttp{take n} function
that returns the first $n$ elements of a list.  The trade-off is quicker
shrinking with a lower probability of finding a smaller counterexample.  For
example, redefining shrink for type \ttp{B} as follows
%

%% \begin{minipage}{\columnwidth}
%% \lstset{ caption={Truncated shrinking function for data type \ttp{B}.}
%%        , label=lst:newshrink
%%        , captionpos=b }
%% \begin{lstlisting}

\begin{figure}[ht]
\begin{code}
len = 10

shrink (B a b c d) = 
  [ B w x y z | w <- tk a, x <- tk b
              , y <- tk c, z <- tk d ]
  where tk x = take len (shrink x)
\end{code}
  \caption{Truncated shrinking function for data type \ttp{B}.}
  \label{lst:newshrink}
\end{figure}

%% \end{lstlisting}
%% \end{minipage}
%
\noindent
controls the exponential blowup of the state-space by taking the first ten
elements of each list, so the cross product is limited to 1000 elements.  The
downside is that potentially smaller counterexamples may be omitted.

The programmer might ask herself: ``What if we omit the need for shrinking
counterexamples altogether?''  SmallCheck is another testing framework for
Haskell that does just this.  SmallCheck is guaranteed to return the a smallest
counterexample, if one exists~\cite{sc}.  SmallCheck does this by enumerating
all possible inputs, ordered from smallest to largest, up to some user-defined
bound.  While SmallCheck is effective for testing many programs and properties
(in accordance with the \emph{small scope hypothesis}~\cite{jackson}),
counterexamples to even relatively simple properties may be infeasible to
discover due to state-space explosion.

With SmallCheck, a property is checked up to a depth set by the user, measuring
how large values are.  For example, for type \ttp{Int}, the depth $d$ is defined
as the integers in the list enumerating from $-d$ to $d$ (i.e.,
\ttp{[(-d)..d]}).  For a product type, SmallCheck must check values produced by
taking the cross-product of the type's fields at a given depth.  In
Figure~\ref{fig:initial}, we have defined instances for generating Lazy
SmallCheck (a more efficient version of SmallCheck) tests by defining instances
of the \ttp{Serial} class.  For the property \ttp{prop} defined above (suitably
redefined to use SmallCheck's types), SmallCheck must check to depth 16 to find
the first counterexample.  Unfortunately, after a couple hours of testing, Lazy
SmallCheck is still checking values at depth six, and the number of tests scales
exponentially with respect to the depth.\footnote{All tests in this paper are
  performed on a four-core Pentium~i7 running at 2.7GHz with 8GB RAM on
  Fedora~16.  This test, and others unless noted, are performed using
  interpreted Haskell under GHC 7.4.2.}

Finally, even if we could overcome the problems described above with QuickCheck
and SmallCheck, we are left with two others.  
\begin{itemize}
  \item First, when a counterexample is discovered to some property, say with
    type
%
\begin{code}
prop :: B -> Bool      
\end{code}
%
\noindent
one might wish to search for different counterexamples to \ttp{prop}.  Ideally,
the testing framework would automatically add a predicate characterizing (and
perhaps generalizing) a counterexample
%
\begin{code}
cex :: B -> Bool      
\end{code}
%
\noindent
as a precondition to \ttp{prop} to define a new property
\begin{code}
prop' b = not (cex b) ==> prop b
\end{code}
%
\noindent
where \ttp{==>} is implication to generate new counterexamples to the same property.

  \item Second, and more generally, a counterexample is usually representative
    of other counterexamples, but that generalization is left to the user to
    discover for herself.  Really, what would be useful would be not just a
    minimal counterexample, but a quantified formula characterizing a set of
    counterexamples.  Such a formula has the added benefit of replacing a
    potentially large sub-term of a composite data type value with a quantifier.
\end{itemize}

\begin{figure}[ht]
\footnotesize
  \begin{center}
    \begin{tabular}{|r||c|c|c|c|}
\hline 
 & QC (none) & QC (10) & QC (20) & SmartCheck \\      
\hline \hline 
Max.  & 0.15s & 21.51s & 125.37s & 1.96s\\
\hline
Mean  & 0.07s & 1.21s & 3.80s & 0.30s\\
\hline
Median & 0.07s & 0.48s & 0.52s & 0.24s\\
\hline
Mean size & 70 & 34 & 32 & 5\\
\hline
    \end{tabular}
  \end{center}
  \caption{Results for 100 tests of the program in Figure~\ref{fig:initial}
    using interpreted Haskell.}
  \label{fig:results}
\end{figure}


\paragraph{SmartCheck}
Motivated by these limitations of QuickCheck and SmallCheck, we have developed
\emph{SmartCheck}.  SmartCheck extends QuickCheck to efficiently and generically
find small counterexamples as well as generalize them as described in the
preceding paragraph.  To motivate the benefit of SmartCheck, consider the table
in Figure~\ref{fig:results}.  The table characterizes testing the program shown
in Figure~\ref{fig:initial} that we have described so far.  Four trials are
presented.  In the first three, we use QuickCheck to generate counterexamples.
The first column contains the results of using QuickCheck without shrinking to
provide a baseline.  The next two columns show the result of using QuickCheck
with definitions of \ttp{shrink} that limit the lists of fields of \ttp{B} to 10
(Figure~\ref{lst:newshrink}) and 20 elements (Figure~\ref{lst:newshrink}, with
\ttp{len = 20}), respectively.  (Recall that with the original definition for
shrink from Figure~\ref{fig:initial}, some counterexamples cause QuickCheck to
not return after hours of computation.)  In each row, we show the maximum, mean,
and median number of seconds required by QuickCheck to generate a counterexample
over 100 runs.  In the last row, we show the mean size of the final value
returned, measured by the taking the cumulative length of the lists of the
fields of \ttp{B}.  Notice that while typically counterexamples are returned in
a fraction of a second, there are outliers taking 21 seconds and 125 seconds for
the two tests.  Both tests roughly cut the size of the counterexample in half;
increasing the number of potential counterexamples can result in a significant
performance penalty for this example without much benefit.

In the last column are the results for SmartCheck.  Not only does it execute
faster than QuickCheck with shrinking, it has significantly smaller outliers.
The most striking difference, however, is the size of the counterexample
returned, reducing on average the size of a counterexample found by QuickCheck
without shrinking by a factor of 14.

The purpose of this paper is to explain first how to efficiently and generically
generate small counterexamples for algebraic data types and second how to
generalize the counterexamples.

\paragraph{Contributions}
This paper, and the library it describes, makes the following contributions:

\begin{itemize}

\item an efficient counterexample reduction strategy for algebraic
  data types;

\item an approach to counterexample generalization, to present the
  user with a formula describing a set of counterexamples to a property;

\item an approach to generically strengthen a property with a precondition that
  characterizes an already-discovered counterexample for the purpose of finding
  new, dissimilar counterexamples;

\item an implementation based on generic programming that is not only more
  efficient that hand-written \ttp{shrink} instances in many cases but that
  generates the instances automatically for the user.

%% \item More generally, this paper is the first to explore the idea of
%%   efficient, generic counterexample reduction and generalization for
%%   functional program testing.
\end{itemize}

%% \noindent Moreover, the inability to quickly uncover understandable
%% counterexamples is a real-world problem with QuickCheck (or other functional
%% languages).  It works great when your tests pass,

\paragraph{Outline.}


%%%%%%%%%%%%%%%%%%%%%%%%%%%%%%%%%%%%%%%%%%%%%%%%


\section{Shrinking Data}
In this section, we describe how to efficiently shrink data values.  Before we
begin, recall the basic approach implemented by the \ttp{shrink} method:
generate a list of values, each of which is smaller than the current
counterexample.  Each of the new values generated has nothing in common with the
original counterexample, since each one is freshly generated by recursive
size-constrained calls to \ttp{arbitrary}.

Consider the problem of finding a small counterexample as a graph search
problem.  The set of vertexes are all possible well-typed values (the same as
the counterexample).  There is an edge from vertex $v$ to $v'$ if $v'$ is
smaller (by some measure).  The goal is to fine a vertex from the set of
smallest counterexamples in the graph.

The approach taken by \ttp{shrink} is the most naive graph search algorithm:
it simply generates a random set of potential vertexes to check in the graph.
However, it does not exploit that the original counterexample \ttp{cex} is
already in a neighborhood of likely counterexamples.  So rather than choose
random vertexes, it may be more efficient to test other values that are closely
related to \ttp{cex}.  By restricting the search to \ttp{cex}'s neighborhood, a
smaller counterexample might be missed, and only a local minimum is returned.
However, with a large state space, finding a global minimum by random sampling
vertexes in the graph is possible but too inefficient, as we saw from the
example in the introduction.

\subsection{Reduction Algorithm Overview}
The algorithm we present for efficiently searching for new counterexamples can
be thought of as a recursive breadth-first traversal over the structure of the
current counterexample value, where the value is represented as a tree.  At each
node, during the traversal, we generate arbitrary \emph{structurally} smaller
sub-values and build a new value from that, leaving the remainder of the tree
unchanged.  Intuitively, by a structurally smaller value, we mean one with fewer
constructors.  We continue until we reach a fixed-point.

To present the algorithm, consider the \ttp{SubTypes} type class below for which
QuickCheck's \ttp{Arbitrary} class is a super-class.\footnote{This presentation
  deviates from the implementation; for example, we elide details regarding
  error handling.}
%
\begin{code}
class Arbitrary a => SubTypes a where
  size    :: a -> Int
  index   :: a -> Int -> Maybe SubVal
  replace :: a -> Int -> SubVal -> a
\end{code}
%
\noindent
\ttp{SubTypes} has three methods.  The \ttp{size} method returns the size of a
value, \ttp{index} returns a sub-value at an breadth-first index in a value, and
\ttp{replace} replaces a sub-value at a particular focus.  If the index is
out-of-bounds (too large or negative), \ttp{index} return \ttp{Nothing}.  The
method \ttp{replace} returns the original value for an out-of-bounds index or an
incorrectly-typed replacement value.  The \ttp{SubVal} type is an existential
data type, used as a generic container for sub-values from a counterexample.
%
\begin{code}
data SubVal = 
  forall a. (Arbitrary a, SubTypes a) => SubVal a
\end{code}
%
\noindent
For example, consider a binary tree type
%
\begin{code}
data Tree = L | B Tree Tree
\end{code}
% 
\noindent
and the value \ttp{tree}, labeled with the indexes of its sub-values
%
\begin{code}
tree = B\sub{0} (B\sub{1} L\sub{3} 
             (B\sub{4} L\sub{6} L\sub{8}))
          (B\sub{2} L\sub{5} L\sub{7})
\end{code}
%
\noindent
Here are example applications of \ttp{SubTypes}; in the following, we show
the indexes with respect to the \ttp{tree} value:
%
\begin{code}
size tree = 9

index tree 4 = (Just . SubVal) (B\sub{4} L\sub{6} L\sub{8})

index tree 12 = Nothing

replace tree 2 (SubVal L) = 
  B\sub{0} (B\sub{1} L\sub{3} 
        (B\sub{4} L\sub{6} L\sub{8}))
     L
\end{code}
%
\noindent
We will sometimes refer to unwrapped value returned by \ttp{index a i} as the
\emph{$i$th sub-value of $a$}, so for example, \ttp{B\sub{4} L\sub{6} L\sub{8}}
is the 4th sub-value of \ttp{tree}.

\begin{figure}[ht]
  \lstset{caption={}
%         ,xleftmargin=5ex,firstnumber=0,numbers=left
         }
  \begin{lstlisting}
reduce :: SubTypes a 
       => a -> (a -> Property) -> IO a
reduce cex prop = reduce' 1
  where
  reduce' idx =
    case index cex idx of 
      Nothing -> return cex
      Just v  -> do
        vs <- newVals v
        case test cex idx vs prop of
          Nothing   -> reduce' (idx+1)
          Just cex' -> reduce cex' prop

newVals :: SubVal -> IO [SubVal]
newVals (SubVal a) = 
  sequence (replicate maxTries s)
  where s =     sizedArbitrary a 
            >>= return . SubVal

test :: SubTypes a 
     => a -> Int -> [SubVal] 
     -> (a -> Property) -> Maybe a
test cex idx vs prop = go vs
  where
  go []      = Nothing
  go (v:vs') = 
    let cex' = replace cex idx v in
    if pass prop cex' then go vs' 
      else Just cex'
  \end{lstlisting}
  \caption{Counterexample reduction algorithm.\label{fig:reduction}}
\end{figure}

Figure~\ref{fig:reduction} shows the reduction algorithm.  The function
\ttp{reduce} takes a counterexample \ttp{cex} and the property \ttp{prop}.  At
the beginning of execution, we assume that a counterexample has been obtained
somehow; our implementation relies on QuickCheck as a ``front-end'' to generate
a counterexample that the algorithm automatically shrinks.  The reduction begins
at the first sub-value of \ttp{cex}, \ttp{v}.  When the index \ttp{idx} becomes
out-of-bounds and returns \ttp{Nothing}, the algorithm terminates.  Otherwise, a
list of new random values are generated.
%
\begin{code}
sizedArbitrary :: SubTypes a => a -> IO a  
\end{code}
%
\noindent
generates a new value \ttp{v'} having the same type as \ttp{v} and strictly
smaller (with respect to the \ttp{size} method) than \ttp{v}.  Just like
QuickCheck's \ttp{arbitrary} method, \ttp{sizedArbitrary} generates successfully
larger counterexamples when generating new values with which to replace a
sub-value.

\ttp{sizedArbitrary} returns a value inside the \ttp{IO} monad since it depends
on a random seed for generation (we could pass in the generator seed, but for
simplicity, we assume one is generated locally).  The constant \ttp{maxTries} is
the maximum number of tries to discover a new counterexample by replacing
\ttp{v} in \ttp{cex} and testing it.  The result of \ttp{pass prop cex'} for
%
\begin{code}
pass :: (a -> Property) -> a -> Bool
\end{code}
%
\noindent
holds if \ttp{cex'} satisfies the property \ttp{prop}.  If no failure is found,
we move to the next sub-value of \ttp{cex} and continue.  However, if a new
smaller counterexample \ttp{cex'} is found, we start a new breadth-first
traversal of \ttp{cex'}, attempting to shrink it further.

\subsection{Reduction Algorithm Details and Optimizations}
The reduction algorithm description above omits some minor details and
optimizations we describe here.

\paragraph{Sum-type counterexample hypothesis}
First, recall that the algorithm begins at the \emph{first} sub-value of the
counterexample rather than the 0th sub-value.  The 0th sub-value of \ttp{cex} is
\ttp{cex} itself.  No invariant of the algorithm would be violated by beginning
with the 0th sub-value; recall that it would generate strictly smaller values
than \ttp{cex}.  However, we begin with a strict sub-value because of a
hypothesis we refer to as the \emph{sum-type counterexample hypothesis}.  The
hypothesis is that in the search space of possible values, if a counterexample
of the form \ttp{C}~$\vec{x}$, where \ttp{C} is some constructor and $\vec{x}$
are its arguments, is discovered, then it is more likely to find another
``nearby'' counterexample of the form \ttp{C}~$\vec{y}$, where $\vec{y}$ are
different arguments to \ttp{C}.  For example, consider a property about
unintended variable capture over a language's parse tree represented by a sum
type with constructors for module imports, function definitions, and
global-variable assignments, respectively.  A function definition counterexample
can only be reduced to smaller function definition counterexamples, the only
construct in which variable capture is possible.

Another way to motivate the hypothesis is that searching the space of possible
counterexamples by generating \emph{any} well-typed value that is strictly
smaller than the original counterexample is something we have seen before: it is
precisely the naive approach to implementing \ttp{shrink}!  

\paragraph{Conditional properties}
A QuickCheck property may be a conditional of the form
%
\begin{code}
prop a = pre a ==> prop' a
\end{code}
%
\noindent
such that \ttp{prop} is true if the precondition \ttp{pre a} is false or
\ttp{prop' a} holds.  In practice, conditionals are used to restrict the set of
values tested; values that fail the precondition are discarded.  Thus, the call
to \ttp{pass} in the reduction algorithm only considers a potential
counterexample \ttp{cex} to have failed if it passes the property's
precondition and fails the consequent.

\paragraph{Sub-value counterexample hypothesis}
Sometimes, a counterexample fails a property due to a sub-value nested deep
inside the counterexample.  The rest of the value is irrelevant.  We call this
the \emph{sub-value counterexample hypothesis.}  Thus, one way to efficiently
search the space of potential counterexamples is to test a counterexample's
(well-typed) sub-values.

For example, consider a simple calculator language containing constant
\ttp{Int}s, addition, and division:
%
\begin{code}
data Exp = C Int
         | Add Exp Exp
         | Div Exp Exp  
\end{code}
%
\noindent
and consider an interpreter for the language that returns \ttp{Nothing} if
division-by-zero is encountered and \ttp{Just} the value otherwise:
%
\begin{code}
eval :: Exp -> Maybe Int
eval (C i) = Just i
eval (Add e0 e1) =
  liftM2 (+) (eval e0) (eval e1)
eval (Div e0 e1) = 
  let e = eval e1 in 
  if e == Just 0 then Nothing 
    else liftM2 div (eval e0) e
\end{code}
%
\noindent
Suppose we restrict the set of programs with the following predicate:
%
\begin{code}
divSubTerms :: Exp -> Bool
divSubTerms (C _)         = True
divSubTerms (Div _ (C 0)) = False
divSubTerms (Add e0 e1)   =    divSubTerms e0 
                            && divSubTerms e1
divSubTerms (Div e0 e1)   =    divSubTerms e0 
                            && divSubTerms e1  
\end{code}
%
and then test the property
%
\begin{code}
div_prop e = divSubTerms e ==> eval e /= Nothing  
\end{code}
%
\noindent
The property \ttp{div\_prop} turns out to be false; for example, 
%
\begin{code}
Div (C 1) (Div (C 0) (C 1))
\end{code}
%
\noindent
is a counterexample.  In the predicate case
%
\begin{code}
divSubTerms (Div _ (C 0)) = False  
\end{code}
%
\noindent
we forgot to recur on the divisor.  However, a typical counterexample returned
by QuickCheck, without \ttp{shrink} implemented, is as follows:
%
\begin{quote}
\ttp{Div (Div (C (-22)) (Div (C 2) (C 13))) (Div (Div (Div (C (-11)) (C 14))
  (Add (Div (C (-4)) (C (-3))) (Div (C (-25)) (C (-25))))) (Add (Add (C 22) (Div
  (C (-30)) (C (-14)))) (C 13)))}
\end{quote}
%
\noindent
To butcher an expression, finding the cause of failure is like searching for a
divide-by-zero in the haystack!  But it turns out that the culprit is a small
sub-value:
%
\begin{code}
Div (C (-22)) (Div (C 2) (C 13))  
\end{code}
%

\begin{figure}
\begin{lstlisting}
reduce0 :: forall a . SubTypes a 
        => a -> (a -> Property) -> IO a
reduce0 cex prop = reduce' 1
  where
  reduce' idx = 
    case index cex idx of 
      Nothing -> return cex
      Just v  -> 
        case testHole v of
          Just v' -> reduce0 v' prop
          Nothing -> do
            vs <- newVals v
            case test cex idx vs prop of
              Nothing   -> 
                reduce' (idx+1)
              Just cex' -> 
                reduce0 cex' prop

  testHole (SubVal a) = 
    case cast a :: Maybe a of
      Nothing -> Nothing
      Just a' -> 
        if pass prop a' then Nothing
          else Just a'
\end{lstlisting}
  \caption{Reduction algorithm with the sub-value counterexample optimization.}
  \label{fig:reduce0}
\end{figure}

Generalizing this scenario, during the reduction algorithm's breadth-first
search through a counterexample \ttp{cex}'s sub-values, we may happen upon a
sub-value \ttp{cex'} that has the same type as \ttp{cex} and fails the property
(while passing any preconditions).  In this case, we can return \ttp{cex'}
directly, and rerun the reduction algorithm on \ttp{cex'}.  In
Figure~\ref{fig:reduce0}, we show an updated reduction algorithm, \ttp{reduce0}
that implements this optimization.  The function \ttp{testHole} tests the
current sub-value and if it fails the property, the function tests the sub-value
directly.

\paragraph{Bounding counterexample exploration}
\todo{limit depth}
\todo{limit size of generated structures}


\paragraph{Base types}






%% For example, consider the following types that give us a binary tree
%% %
%% \begin{code}
%% data T = N Int
%%        | B T T
%% \end{code}
%% %
%% \noindent
%% and suppose that the property is that the sum of the leaves is less than 10.
%% Starting from the following counterexample
%% %
%% \lstset{caption={A counterexample value.}
%%        ,label=lst:val}
%% \begin{lstlisting}
%% B (B (N 2) (N 3)) 
%%   (B (N 4) (B (N 5) (N 6)))
%% \end{lstlisting}
%% %
%% \noindent
%% Consider the sub-value \ttp{B (N 4) (B (N 5) (N 6))}.  Some example values of
%% structurally smaller well-typed values include the following:
%% %
%% \begin{code}
%% N 2
%% B (N 8) (N 7)
%% B (N 0) (N 1)
%% \end{code}
%% %
%% \noindent
%% Replacing the sub-value, we get the following new values, each of which is
%% smaller than the value in Listing~\ref{lst:val}:
%% %
%% \begin{code}
%% B (B (N 2) (N 3)) 
%%   (N 2)

%% B (B (N 2) (N 3)) 
%%   (B (N 8) (N 7))

%% B (B (N 2) (N 3)) 
%%   (B (N 0) (N 1))
%% \end{code}

In this work, we limit ourselves to algebraic data types.

\todo{GADTs?}

\section{Counterexample Generalization}


\section{Implementation}

\todo{generics}


\section{Experiments}

%--------------------------------------------------------------------------------


%--------------------------------------------------------------------------------
\section{Related Work}
\label{sec:related}

\todo{shrink paper rehger pointed to ``Automatic isolation of compiler errors'' from 1994}

\cite{}

%--------------------------------------------------------------------------------
\section{Conclusions}
\label{sec:conclusions}


\section*{Acknowledgements}
\todo{Thank Rehger for comments/pointing to paper.}

%% \balancecolumns

\bibliographystyle{abbrvnat}
\bibliography{paper}

\end{document}

